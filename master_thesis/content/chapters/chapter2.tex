\chapter{State of the art}
\label{ch:state-of-the-art}

During the last few years, the problem was studied by a few researchers around the world.\cite{Krenn_2017,bogdanov,chandran,chandran2023graphtheoretic}
While none of them could find any constant upper bound on the weighted matching index of perfectly monochromatic graphs (defined in definitions~\ref{def:weighted_matching_index} and~\ref{def:perfectly_monochromatic_graph}), some special cases of the conjecture were proven to be true.
Furthermore, non-constant bounds could also be found in more general cases of experiment graphs.
This section aims to present these results.
But before getting started, it could be interesting to introduce some useful definitions and tools that will be used a lot during the different seen proofs.

\begin{definition}[(Non-)redundant experiment graph]
    \label{def:redundant-experiment-graph}
    In the context of this master thesis, an experiment graph $G_k^w$ as defined in definition~\ref{def:experiment_graph} is said to be \textit{non-redundant} if it satisfies all the following properties.

    \begin{itemize}
        \item $\forall e \in E(G), w(e) \neq 0$
        \item $\forall e \in E(G), \exists$ A perfect matching $M$ such that $e \in M$
        \item $\forall (v_1, v_2) \in (V(G))^2, v_1 \neq v_2,$ and $\forall$ colour pair $(r, g)$, $\exists$ at most one unique edge $e$ between $v_1$ and $v_2$ such that $k(e, v_1) = r$ and $k(e, v_2) = g$, using notations introduced in definition~\ref{def:mixed_edge_coloured_graph}.
    \end{itemize}

    Otherwise, it is said to be \textit{redundant}.
\end{definition}


\begin{definition}[Non-redundant induced graph]
    \label{def:non_redundant_induced_subgraph}
    Let $G_k^w$ be a redundant experiment graph.
    Then its \textit{non-redundant-induced graph} is the graph built by doing successively the following operations.

    \begin{itemize}
        \item $\forall e \in E(G)$ such that $w(e) = 0$, delete $e$.
        \item $\forall e \in E(G)$ that does not belong to any perfect matching $M$ of $G_k^w$, delete $e$.
        \item $\forall (v_1, v_2) \in (V(G))^2, v_1 \neq v_2$ And $\forall$ colour pair $(r, g)$, if there are multiple edges $e_1, e_2, \dots, e_m$ such that
        
        \begin{center}
            $\forall i \in \{1, \dots, m\}$, $k(e_i, v_1) = r$ and $k(e_i, v_2) = g$
        \end{center}
        Then, replace them by one single edge $e = (v_1, v_2)$ that has the following properties.
        
        \begin{center}
            $k(e, v_1) = r$ And $k(e, v_2) = g$ \\
            $w(e) = \sum\limits_{i = 1}^m w(e_i)$
        \end{center}
        
    \end{itemize}
    
\end{definition}

This definition and denomination is entirely motivated by the following observation.

\begin{observation}[Non-redundancy is enough]
    \label{obs:non_redundancy_enough}
    Let $G_k^w$ be a redundant experiment graph, and let ${G'}_{k'}^{w'}$ be its non-redundant induced graph. Then $\Tilde{c}(G, k, w) = \Tilde{c}(G', k', w')$.
\end{observation}

\begin{proof}
    To prove this last observation, the procedure will be to show that each of the transformations that was applied on $G_k^w$ did not change its weighted matching index.

    \begin{enumerate}
        \item \textbf{Transformation 1 :} $\forall e \in E(G)$ such that $w(e) = 0$, delete $e$. \\
            Let $e = (v_1, v_2)$ be a zero-weighted edge of $G_k^w$. At most, $e$ can contribute to the weights of all the feasible vertex colourings $\kappa$ such that $\kappa(v_1) = k(e, v_1)$ and $\kappa(v_2) = k(e, v_2)$. But since $w(e) = 0$, then all the perfect matchings $M$ such that $e \in M$ have a weight $w(M) = 0$. Therefore, $e$'s contribution to $\kappa$ is null, and removing it doesn't change anything to the feasibility of $\kappa$.

        \item \textbf{Transformation 2 :} $\forall e \in E(G)$ that do not belong to any perfect matching $M$ of $G_k^w$, delete $e$. \\
            Since $e$ does not belong to any perfect matching $M$, its weight can not contribute to any feasible vertex colouring $\kappa$. Removing it has therefore no impact.

        \item \textbf{Transformation 3 :} $\forall (v_1, v_2) \in (V(G))^2, v_1 \neq v_2$ and $\forall$ colour pair $(r, g)$, if there are multiple edges $e_1, e_2, \dots, e_m$ such that
        \begin{center}
            $\forall i \in \{1, \dots, m\}$, $k(e_i, v_1) = r$ and $k(e_i, v_2) = g$
        \end{center}
        Then, replace them by one single edge $e = (v_1, v_2)$ that has the following properties.
        \begin{center}
            $k(e, v_1) = r$ And $k(e, v_2) = g$ \\
            $w(e) = \sum\limits_{i = 1}^m w(e_i)$
        \end{center}

            Indeed, let $\kappa$ be a feasible vertex colouring such that $\kappa(v_1) = r$ and $\kappa(v_2) = g$. We need to introduce some new notations for this specific part of the proof, which are inspired of the notations we previously defined in definition \ref{def:induced_vertex_colouring} of an induced vertex colouring.

            \begin{itemize}
                \item $\mathcal{M}_\kappa$ Is the set of all perfect matching $M$ of $G$ that induce a vertex colouring $\kappa$ on $G$.
                \item $\mathcal{M}_\kappa^{e_i}$ Is the set of perfect matchings $M$ of $G$ that induce a vertex colouring $\kappa$ on $G$ such that $e_i \in M$.
                \item $\mathcal{M}_\kappa^{*}$ Is the set of perfect matchings $M$ of $G$ that have no edges from $\{e_1, \dots, e_m\}$.
                \item $\mathcal{M'}_\kappa$ Is the set of all perfect matching of $G'$ that induce a vertex colouring $\kappa$ on $G'$.
                \item $\mathcal{M'}_\kappa^{e}$ Is the set of perfect matchings of $G'$ that induce a vertex colouring $\kappa$ on $G'$ such that $e_i \in M$.
                \item $\mathcal{M'}_\kappa^{*} = \mathcal{M}_\kappa^{*}$ Is the set of perfect matchings $M$ of $G'$ that have do not contain $e$.
            \end{itemize}

            Using these new notations, and by the definition \ref{def:vertex_colouring_weight} of the weight of a vertex colouring, 

            \begin{center}
                $\begin{array}{r c l}
                    w(\kappa) & = & \sum\limits_{M \in \mathcal{M}_\kappa} w(M)  \\
                              & = & \sum\limits_{i = 1}^m \sum\limits_{M \in \mathcal{M}_\kappa^{e_i}} w(M) + \sum\limits_{M \in \mathcal{M}_\kappa^{*}} w(M) \\
                              & = & \sum\limits_{i = 1}^m w(e^i) \sum\limits_{M \in \mathcal{M}_\kappa^{e_i}} w(M \backslash e^i) + \sum\limits_{M \in \mathcal{M}_\kappa^{*}} w(M) \\
                              & = & w(e) \sum\limits_{M \in \mathcal{M'}_\kappa^{e}} w(M \backslash e) + \sum\limits_{M \in \mathcal{M'}_\kappa^{*}} w(M) \\
                              & = & \sum\limits_{M \in \mathcal{M'}_\kappa} w(M)
                \end{array}$
            \end{center}

            This result proves the transformation 3 did not have any influence on the weight of all induced vertex colourings.
    \end{enumerate}
\end{proof}

This last observation is convenient since it allows researchers to focus on non-redundant experiment graphs to prove bounds on their matching index, and these bounds are still valid in redundant experiment graphs without any loss of generality. An example of such transformation is shown in figure \ref{fig:non_redundant_induced_graph}.

\begin{figure}[H]
    \ctikzfig{figures/state_of_the_art/non_redundant_induced_graph}
    \caption{In this figure, $G_k^w$ is an experiment graph and ${G'}_{k'}^{w'}$ is its non-redundant induced graph. In this particular case, it consisted in doing the following operations. First of all, the $0$-weighted edge between $v_3$ and $v_6$ is removed. Secondly, the edge $\{v_1, v_3\}$ is also deleted because it does not belong to any perfect matching (indeed, including it in a perfect matching $M$ would prevent $M$ to cover $v_2$). Lastly, the two edges between $v_4$ and $v_6$ are combined in one single edge, since they have the same colour at each endpoint. The reader can verify that the resulting graph ${G'}_{k'}^{w'}$ has a weighted matching index $\Tilde{c}(G', k', w') = 2$. Therefore, $\Tilde{c}(G, k, w) = 2$ by observation \ref{obs:non_redundancy_enough}.}
    \label{fig:non_redundant_induced_graph}
\end{figure}


\section{Special cases that were already proven}

\subsection{Restrictions on the weights}

\begin{lemma}[Real, positive weights]
    \label{lem:real_pos_weights}
    Let $G_k^w$ be a perfectly monochromatic graph which has only positive, real weights. If $G$ is isomorphic to $K_4$, then $\Tilde{c}(G, k, w) \leq 3$. Otherwise, $\Tilde{c}(G, k, w) \leq 2$.
\end{lemma}

\begin{proof}
    This is a direct result from Bogdanov's proof, described in theorem \ref{thm:bogdanov}. \cite{bogdanov}. Because all the weights of $G_k^w$ are real, positive numbers, it means that the weight of any perfect matching is positive. Therefore, the weight of any feasible vertex colouring is positive by definition \ref{def:vertex_colouring_weight}. But all the non-monochromatic feasible vertex colourings in a perfectly monochromatic graph should have a weight of 0 by definition \ref{def:monochromatic_graph}. It follows that $G_k^w$ has no non-monochromatic perfect matching. Then $\Tilde{c}(G, k, w) = c(G, k)$.
\end{proof}


\subsection{Restrictions on the matching index}

In 2022, Gajjala and Chandran analysed the Krenn's conjecture by separating the graphs in different subclasses according to their matching index (see definition \ref{def:matching_index}). Here are their results.

\begin{lemma}[Graphs with a matching index of 0]
    \label{lem:if_c_is_0}
    If $G$ is a a graph with a matching index of 0, then the Krenn's conjecture is true for $G$ and $\Tilde{c}(G) = c(G) = 0$.
\end{lemma}

\begin{proof}
    Let $G$ be a a graph with a matching index of 0. Then, $G$ has no perfect matchings (otherwise it would be feasible to colour all the edges of $G$ in the same colour and find a monochromatic perfect matching, which contradicts the fact that $c(G) = 0$ by definition \ref{def:matching_index} of a matching index). Because it has no perfect matchings, there is no mixed-edge colouring $k$ and weight attribution $w$ such that $G_k^w$ has a feasible monochromatic vertex colouring. It follows that $\Tilde{c}(G) = 0$.
\end{proof}

\begin{lemma}[Graphs with a matching index of 2]
    \label{lem:if_c_is_2}
    If $G$ is a a graph with a matching index of 2, then the Krenn's conjecture is true for $G$ and $\Tilde{c}(G) = c(G) = 2$.
\end{lemma}

This last lemma is proved by Chandran and Gajjala in \cite{chandran}. The proof is not trivial and uses some very interesting observations about the structure of a perfectly monochromatic graph with $\Tilde{c}(G) = 2$. The reader is highly encouraged to have a look at it in the original paper in order to believe this claim. Nevertheless, the proof will not be discussed here since it would be a bit redundant. Thanks to their proof, Chandran and Gajjala could pose the following theorem, which is a summary of the 2 last lemmas \ref{lem:if_c_is_0} and \ref{lem:if_c_is_2}.

\begin{theorem}[Chandran - Gajjala's theorem]
    \label{thm:c_not_1}
    Let $G$ be a perfectly monochromatic graph. If $c(G) \neq 1$, then the Krenn's conjecture is true and $\Tilde{c}(G) = c(G)$.
\end{theorem}

Having that last theorem, it would be a natural question to ask ourselves if, for any graph $G$, $c(G) = \Tilde{c}(G)$. Unfortunately, it is not the case, since a counter-example was found by Chandran and Gajjala. \cite{chandran}

\begin{observation}
    \label{obs:c_not_c_tilde}
    There exists a graph $G$ that satisfies $c(G) = 1$ and $\Tilde{c}(G) = 2$. This counter example is shown in figure \ref{fig:proof_c_not_c_tilde}.
\end{observation}

\begin{figure}[H]
    \ctikzfig{figures/state_of_the_art/special_cases/c_not_c_tilde}
    % TODO : there's an error here and in the other similar figure
    \caption{Example of graph $G$ that has a matching index of $1$ and a weighted matching index of $2$. On this figure, the edge-colouring $k$ is an example of edge colouring that gives $c(G, k) = 1$. Furthermore, the mixed-edge colouring $k'$ and the weight attribution $w$ are examples of mixed-edge colourings and weights attributions that results in $\Tilde{c}(G, k', w) = 2$. Another example of mixed-edge colouring and weight attribution that gives the same matching index on $G$ is available in figure \ref{fig:perfectly_mono}. It has still to be shown that it is impossible to find other edge-colourings / weight attributions that would lead to bigger (weighted) matching indexes. This proof is not presented here, but is available in Chandran and Gajjala's original paper. \cite{chandran}}
    \label{fig:proof_c_not_c_tilde}
\end{figure}


\section{Known non-constant bounds on the weighted matching index}

Despite the failure to find any constant bound on the weighted matching index of experiment graphs up to now, some interesting non-constant bounds were nevertheless found. Chandran and Gajjala discovered in \cite{chandran} two interesting bounds in terms of minimum degree and edge connectivity that I considered to be relevant to rewrite here. The 2 following theorems are due to them.

\begin{lemma}[Upper bound in terms of minimum degree]
    \label{lem:bound_min_degree}
    Let $G$ be a graph, and $\delta(G)$ be its minimum degree as defined in definition \ref{def:degree}. Then $\Tilde{c}(G) \leq \delta(G)$.
\end{lemma}

\begin{proof}
    Let $G_k^w$ be a perfectly monochromatic graph such that $\Tilde{c}(G, k, w) = \Tilde{c}(G)$. It is known by definition \ref{def:weighted_matching_index} of the weighted matching index that $G_k^w$ has at least one monochromatic perfect matching for $\Tilde{c}(G)$ different colour classes. In the context of this proof, these perfect matchings are denoted $M_1, M_2, \dots, M_{\Tilde{c}(G)}$. Because these perfect matchings are of different colours, they can not share any edge.\\
    
    Let $v \in V(G)$ be a vertex of minimum degree $d(v) = \delta(G)$. This vertex must be covered by $M_1, M_2, \dots, M_{\Tilde{c}(G)}$. $M_1$ Covers it through an edge $e_1$, $M_2$ through an edge $e_2$, $\dots$, $M_{\Tilde{c}(G)}$ through an edge $e_{\Tilde{c}(G)}$ (as shown in figure \ref{fig:proof_min_degree}). Then $v$ has at least $\Tilde{c}(G)$ outgoing edges, and $d(v) = \delta(G) \geq \Tilde{c}(G)$.
\end{proof}

\begin{figure}[H]
    \ctikzfig{figures/state_of_the_art/known_bounds/proof_min_degree}
    \caption{Visualisation of the proof that $\Tilde{c}(G) \leq \delta(G)$ for every perfectly monochromatic graph $G$.}
    \label{fig:proof_min_degree}
\end{figure}

\begin{theorem}[Upper bound in terms of edge connectivity]
    \label{thm:bound_edge_connectivity}
    Let $G$ be a graph, and $\lambda(G)$ be its edge connectivity as defined in definition \ref{def:edge_connectivity}. Then $\Tilde{c}(G) \leq \lambda(G).$
\end{theorem}

The proof of this last theorem is not trivial and uses some concepts that need to be properly defined here. It was first described by Chandran and Gajjala in \cite{chandran}, and we are using the same concepts than them here.

\begin{definition}[Redundant edge]
    \label{def:redundant_edge}
    An edge $e$ from a graph $G$ is said to be \textit{redundant} if it does not belong to any perfect matching of $G$.
\end{definition}

\begin{definition}[Redundant colour class]
    \label{def:redundant_colour_class}
    A colour class $r$ from an edge colouring $k$ of a graph $G$ is said to be \textit{redundant} if there are no monochromatic perfect matching of colour $r$ in $G_k$.
\end{definition}

\begin{definition}[Redundant mixed-colour class]
    \label{def:redundant_mixed_colour_class}
    A mixed-colour class $(r, b)$ from an edge-colouring $k$ of a graph $G$ is said to be \textit{redundant} if at least one of the 2 following conditions is true.
    \begin{center}
        $\left\{
        \begin{array}{l}
            r \mbox{ is a redundant colour class} \\
            b \mbox{ is a redundant colour class}
        \end{array}
        \right.$
    \end{center}
\end{definition}

\begin{proof}[Proof of Theorem \ref{thm:bound_edge_connectivity}]
    This theorem is proven by contradiction. Using the notations introduced in definitions \ref{def:weighted_matching_index} and \ref{def:edge_connectivity}, let assume there exists a graph $G''$ such that $\Tilde{c}(G'') \geq \lambda(G'') + 1$. By removing all the redundant edges (defined in definition \ref{def:redundant_edge}) of $G''$, it results in a graph $G'$ with $\Tilde{c}(G'') = \Tilde{c}(G')$. This manipulation preserves the weighted matching index of the graph.
    
    \begin{center}
        $\Tilde{c}(G') \geq \lambda(G'') + 1$
    \end{center}
    
    Therefore, there exists a mixed-edge colouring $k'$ and a weight attribution $w'$ such that
    
    \begin{center}
        $\Tilde{c}(G') = \Tilde{c}(G', k', w') \geq \lambda(G'') + 1$
    \end{center}
    
    From ${G'_{k'}}^{w'}$, let apply a new transformation that removes all edges that belong to a redundant colour class and all edges that belong to a redundant mixed-colour class (defined in definitions \ref{def:redundant_colour_class} and \ref{def:redundant_mixed_colour_class}), obtaining $G_k^w$. It is easy to see that, if ${G'_{k'}}^{w'}$ was perfectly monochromatic, $G_k^w$ is still perfectly monochromatic. Indeed, by removing all the edges from a redundant colour class, no monochromatic perfect matching is removed (since the colour class was redundant) and, if there were (non-monochromatic) perfect matchings containing edges of that colour class, they are all destroyed, so there is no induced vertex colouring of $G$ anymore that has vertices of the removed colour class - then it's not needed anymore to worry about the weight of such vertex colourings. \\
    
    $\Tilde{c}(G)$ Did not decrease, because no non-redundant colour class was removed from ${G'_{k'}}^{w'}$, and did not increase, since removing edges can not create new perfect matchings. So again :
    
    \begin{center}
        $\Tilde{c}(G) = \Tilde{c}(G') = \Tilde{c}(G'')$
    \end{center}
    
    We also notice that the edge-connectivity of a graph can not increase when we remove edges. Therefore
    
    \begin{center}
        $\begin{array}{l c l}
            \Tilde{c}(G) & \geq & \lambda(G'') + 1 \\
                         & \geq & \lambda(G') + 1 \\
                         & \geq & \lambda(G) + 1
        \end{array}$
    \end{center}
    
    By definition \ref{def:edge_connectivity} of the edge-connectivity of a graph, $G$ can be cut in 2 disconnected components $S$ and $S'$ by removing $\lambda(G)$ edges. Note that, if $G$ has perfect matchings, $|V(G)|$ is an even number and therefore $|S|$ and $|S'|$ are of the same parity.
    
    \begin{enumerate}
        \item 
            \textbf{If $|S|$ and $|S'|$ are odd} then, for all monochromatic perfect matching $M$, $M$ contains at least one crossing edge, i.e. an edge with one endpoint in $S$ and the other endpoint in $S'$ (see figure \ref{fig:proof_lambda_odd}).
            
            \begin{figure}[H]
                \ctikzfig{figures/state_of_the_art/known_bounds/proof_lambda_odd}
                \caption{Existence of a crossing edge between $S$ and $S'$ in a perfect matching $M$ of $G$ if $|S|$ and $|S'|$ are odd.}
                \label{fig:proof_lambda_odd}
            \end{figure}
            
            Let $E(S, S')$ be the set of crossing edges from $S$ to $S'$. The maximum number of different colour classes that can be found on $E(S, S')$ is $\lambda(G)$ (otherwise more edges wold be needed). The conclusion is that the monochromatic perfect matchings of $G_k^w$ can be of a most $\lambda(G)$ different colours. Hence,
            
            \begin{center}
                $\Tilde{c}(G'') = \Tilde{c}(G) \leq \lambda(G) \leq \lambda(G'')$
            \end{center}
            
            This forms a contradiction with the statement that $\Tilde{c}(G'') \geq \lambda(G'') + 1$.
            
        \item
            \textbf{If $|S|$ and $|S'|$ are even} then let's separate all the colour classes of $G_k^w$ in
            
            \begin{center}
                $\left\{ \begin{array}{l c l c l}
                    [r]             & = & \{1, 2, \dots, r\}              & = & \mbox{all the colours such that none of the monochromatic} \\
                                    &   &                                 &   & \mbox{perfect matchings of these colours intersects } E(S, S') \\ 
                    {[r + 1, r + r']} & = & \{r + 1, r + 2, \dots, r + r'\} & = & \mbox{all the colours such that there exists a monochromatic} \\
                                    &   &                                 &   & \mbox{perfect matching of this colour intersecting } E(S, S')
                \end{array} \right.$
            \end{center}
            
            Clearly, $\Tilde{c}(G'') = \Tilde{c}(G) = r + r'$. For all colours $i \in [r + 1, r + r']$, there exists a monochromatic perfect matching $M$ which intersects $E(S, S')$. And since $|S|$ and $|S'|$ are even, $M$ contains at least 2 edges from $E(S, S')$, as shown in figure \ref{fig:proof_lambda_even}.
            
            \begin{figure}[H]
                \ctikzfig{figures/state_of_the_art/known_bounds/proof_lambda_even}
                \caption{Existence of $2$ crossing edges in a crossing perfect matching $M$ of $G$ if $|S|$ and $|S'|$ are even numbers.}
                \label{fig:proof_lambda_even}
            \end{figure}
            
            Then there exist at least 2 edges of colour $i$ in $E(S, S')$ for all $i \in [r + 1, r + r']$. It follows that $\lambda(G'') \geq \lambda(G) \geq 2 \cdot r'$.
            
            \begin{enumerate}
                \item 
                    \textbf{If $r \leq 1$ and $r' \geq 1$}, then 
                    
                    \begin{center}
                        $\begin{array}{l c l}
                            \Tilde{c}(G'')  & =    & r + r' \\
                                            & \leq & 2 \cdot r' \\
                                            & \leq & \lambda(G'')
                        \end{array}$
                    \end{center}
                    
                \item 
                    \textbf{If $r \leq 1$ and $r' = 0$}, then it is trivial because $G$ is connected.
                    
                    \begin{center}
                        $\begin{array}{lcl}
                            \lambda(G'')  & \geq & 1 \\
                                          & \geq & r + r' \\
                                          & =    & \Tilde{c}(G'')
                        \end{array}$
                    \end{center}
                    
                \item
                    \textbf{Else, $r \geq 2$}. Then one can pick 2 colours from $[r]$ (let say $1$ and $2$).
                    
                    \begin{claim}
                        There should be at least 2 mixed edges of colour $(1, 2)$ in $E(S, S')$.
                    \end{claim}
                    
                    \begin{proof}[Proof of the claim]
                        By contradiction, suppose not. Let's consider 2 monochromatic perfect matchings $M_1$ and $M_2$ of colour $1$ and $2$ that induce the monochromatic vertex colourings $1_{V(G)}$ and $2_{V(G)}$ respectively. $M_1$ And $M_2$ do not have any edge in $E(S, S')$. In general in the context of this proof, $i_A$ will denote the monochromatic vertex colouring of colour $i$ of the vertices in $A$. Because $G_k^w$ is perfectly monochromatic, the weights of $1_{V(G)}$ and $2_{V(G)}$ must be $1$. Since $1, 2 \in [r]$, there are no normal edges of colour $1$ nor $2$ in $E(S, S')$. Then we have the following relations, where $w(i_A)$ denotes the weight of the the monochromatic vertex colouring of colour $i$ on the subgraph $A$.
                        
                        \begin{center}
                            $\begin{array}{l c l c l}
                                w(1_{V(G)}) & = & w(1_S) \cdot w(1_{S'}) & = & 1 \\
                                w(2_{V(G)}) & = & w(2_S) \cdot w(2_{S'}) & = & 1
                            \end{array}$
                        \end{center}
                        
                        Therefore, $w(1_S)$, $w(1_{S'})$, $w(2_S)$, $w(2_{S'})$ must be non-zeros. \\

                        Now let's consider a non-monochromatic vertex colouring $\kappa$ in which $S$ is coloured $1$ and $S'$ is coloured $2$. $\kappa$ Is feasible because we can build a perfect matching by taking all the edges from $M_1$ on $S$ and all the edges from $M_2$ on $S'$. Because $G_k^w$ is perfectly monochromatic, $w(\kappa)$ is 0 by definition \ref{def:perfectly_monochromatic_graph}. Also, since $|S|$ and $|S'|$ are even, every perfect matching contains an even number of edges from $E(S, S')$. But by assumption, there is at most one crossing edge of colour $\{1, 2\}$ and no crossing edges of colour $1$ nor $2$. In conclusion, no perfect matching that induces $\kappa$ can contain an edge from $E(S, S')$. Then
                        
                        \begin{center}
                            $w(\kappa) = w(1_S) \cdot w(2_{S'}) = 0$
                        \end{center}
                        
                        But $w(1_S)$ and $w(2_{S'})$ are non-zeros, so this is impossible and creates a contradiction. This proves the claim.
                    \end{proof}

                    \textbf{We proved our claim.} Therefore, for a pair of colours $i, j \in [r]$, there should be at least 2 mixed edges of colour $\{i, j\}$ in $\{S, S'\}$.
                    
                    \begin{center}
                        Minimum number of edges in $E(S, S') = 2 {r \choose 2} + 2 \cdot r' \leq \lambda(G'')$
                    \end{center}
                    
                    Finally, it means that
                    
                    \begin{center}
                        $\begin{array}{lcl}
                            \Tilde{c}(G'') & =    & r + r' \\
                                           & \leq & 2 {r \choose 2} + 2 \cdot r' \\
                                           & \leq & \lambda(G'')
                        \end{array}$
                    \end{center}
                    
                    And this ends the proof.
            \end{enumerate}
    \end{enumerate}
\end{proof}

This proof was interesting to rewrite here since it uses some very interesting reasoning about the structure of perfectly monochromatic graphs and the definitions of their weights. Some similar reasoning is sometimes reused in this master thesis in order to prove my own new results. \\

Last but not least, Chandran and Gajjala showed more recently an upper bound on the weighted matching index of graphs in term of their number of vertices. \cite{chandran2023graphtheoretic}

\begin{theorem}[Upper bound in terms of number of vertices]
    \label{thm:bound_num_vertices}
    Let $G_k^w$ be an experiment graph that has $n$ vertices. Then
    \begin{center}
        $\Tilde{c}(G, k, w) \frac{n}{\sqrt{2}}$
    \end{center}
\end{theorem}

The proof of theorem \ref{thm:bound_num_vertices} is detailed in Chandran and Gajjala's publication \cite{chandran2023graphtheoretic}.


\section{Computational approach}