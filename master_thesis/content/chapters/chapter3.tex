\chapter{New theoretical contributions}
\label{ch:new-contributions}

There are $2$ intuitive ways to tackle the problem of the Krenn's conjecture.
The first way, done in this chapter, is to use a theoretical approach to find interesting properties about perfectly monochromatic graphs.
In this approach, I hope that the conjecture is true and try to prove it in some very restricted cases.
Also, I present $2$ different ways of thinking to the problems by showing an equivalency between $2$ cases in the subsection~\ref{sec:problem-reduction}.\\

The second approach is a computational approach, and is presented in chapter~\ref{ch:computational-approach}.
The goal of this second approach will be to generate a big number of random perfectly monochromatic experiment graphs, and extract conclusions from these data.
To do so, it will present a new tool I developed, called EGPI, that aims to perform such experiments.


\section{Problem reduction}
\label{sec:problem-reduction}

\begin{lemma}[PM graphs with integer weights]
    Let $G_k^w$ be a perfectly monochromatic graph that has only edge weights included in $\mathbb{Z}$.
    For all upper bounds $\beta \in \mathbb{N}$, if the following conjecture is true:
    \begin{itemize}
        \item $\forall {G'_{k'}}^{w'}$ perfectly monochromatic graphs that have only weights included in $\{-1, 1\}$, $\Tilde{c}(G', k', w') \leq \beta$.
    \end{itemize}
    then we have also $\Tilde{c}(G, k, w) \leq \beta$.
\end{lemma}

\begin{proof}
    Let $G_k^w$ be a perfectly monochromatic graph that has only edge weights included in $\mathbb{Z}$.
    We will show that we can build a perfectly monochromatic graph that has only weights included in $\{-1, 1\}$ and that has the same weighted matching index than $G_k^w$. \\

    First, we choose an edge in $G_k^w$ that has a weight different from $1$ nor $-1$ (if such an edge doe not exist, we are done).
    We replace this edge by $|w(e)|$ parallel edges of the same colour, and that have all a weight of $\frac{w(e)}{|w(e)|}$ (1 if $w(e) > 0$, -1 if $w(e) < 0$).
    Remember this can be done because experiment graphs allow multi edges by definition~\ref{def:experiment_graph}.
    This creates a new graph ${G'_{k'}}^{w'}$.
    That process is illustrated in figure~\ref{fig:demo_integers}.\\

    \begin{figure}[H]
        \ctikzfig{figures/new_results/problem_reduction/demo_integers}
        \caption{Illustration of the transformation from an experiment graph with integer weights to an experiment graph with weights included in $\{-1, 1\}$.}
        \label{fig:demo_integers}
    \end{figure}

    Let $\kappa$ be a feasible vertex colouring of $G_k^w$.
    What is the weight of $\kappa$ in $G_k^w$ ? To express it, we will denote by
    \begin{center}
        $\left\{
            \begin{array}{ll}
                M_{\kappa}          & \mbox{the set of perfect matchings of } G_k^w \mbox{ that induce the vertex colouring } \kappa \\
                M_{\kappa}^e        & \mbox{the set of perfect matchings of } G_k^w \mbox{ that induce the vertex colouring } \kappa \mbox{ and contain } e \\
                M_{\kappa}^{\neg e} & \mbox{the set of perfect matchings of } G_k^w \mbox{ that induce the vertex colouring } \kappa \mbox{ and do not contain } e
            \end{array}
        \right.$
    \end{center}

    The weight of $\kappa$ in $G_k^w$ is

    \begin{center}
        $\begin{array}{lclcl}
            w(\kappa \mbox{ in } G_c^w)
                & = & \sum\limits_{M \in M_{\kappa}} w(M) \\
                & = & \sum\limits_{M \in M_{\kappa}^{\neg e}} w(M) & + & \sum\limits_{M \in M_{\kappa}^{e}} w(M)
        \end{array}$
    \end{center}

    The next step, which is the heart of the proof, consists of computing the weight of the vertex colouring $\kappa$ in ${G'_{k'}}^{w'}$.
    We will denote by

    \begin{center}
        $\left\{
            \begin{array}{ll}
                M'_{\kappa}            & \mbox{the set of perfect matchings of } {G'_{k'}}^{w'} \mbox{ that induce the vertex colouring } \kappa \\
                {M'_{\kappa}}^e        & \mbox{the set of perfect matchings of } {G'_{k'}}^{w'} \mbox{ that induce the vertex colouring } \kappa \\
                                   & \mbox{and contain an edge that was derived from } e \\
                {M'_{\kappa}}^{\neg e} & \mbox{the set of perfect matchings of } {G'_{k'}}^{w'} \mbox{ that induce the vertex colouring } \kappa \\
                                   & \mbox{and does not contain an edge that was derived from } e
            \end{array}
        \right.$
    \end{center}

    In addition to these concepts, for every perfect matching $M \in M_{\kappa}^e$, let $\mathcal{M}'(M)$ be the set of corresponding perfect matchings $M'$ in ${M'_{\kappa}}^e$.
    In other words, $\mathcal{M}'(M)$ denotes the perfect matchings that are the same as $M$ on every edge except $e$, and that contain one of the edges that were added when $e$ was removed.
    It follows that, for every $M \in M_{\kappa}^e$, $|\mathcal{M}'(M)| = w(e)$.
    Also, given $M \in M_{\kappa}^e$, for each perfect matching $M' \in \mathcal{M}'(M)$, $w(M') = \frac{w(M)}{|w(e)|}$.
    Finally, we notice the following relations between the different sets we defined :

    \begin{center}
        $\left\{
            \begin{array}{lcl}
                {M'_{\kappa}}^{\neg e} & = & M_{\kappa}^{\neg e} \\
                {M'_{\kappa}}^e        & = & \bigcup\limits_{M \in M_{\kappa}^e} \mathcal{M}'(M)
            \end{array}
        \right.$
    \end{center}

    Having all these observations in mind, we can now compute the weight of $\kappa$ in ${G'_{k'}}^{w'}$.

    \begin{center}
        $\begin{array}{lclcl}
            w(\kappa \mbox{ in } {G'_{k'}}^{w'})
                & = & \sum\limits_{M' \in M'_{\kappa}} w(M') \\
                & = & \sum\limits_{M' \in {M'_{\kappa}}^{\neg e}} w(M') & + & \sum\limits_{M' \in {M'_{\kappa}}^{e}} w(M') \\
                & = & \sum\limits_{M \in {M_{\kappa}}^{\neg e}} w(M)    & + & \sum\limits_{M \in M_{\kappa}^e} \left( \sum\limits_{M' \in \mathcal{M}'(M)} w(M') \right) \\
                & = & \sum\limits_{M \in {M_{\kappa}}^{\neg e}} w(M)    & + & \sum\limits_{M \in M_{\kappa}^e} \left( \sum\limits_{M' \in \mathcal{M}'(M)} \frac{w(M)}{|w(e)|} \right) \\
                & = & \sum\limits_{M \in {M_{\kappa}}^{\neg e}} w(M)    & + & \sum\limits_{M \in M_{\kappa}^e} \left( |w(e)| \frac{w(M)}{|w(e)|} \right) \\
                & = & \sum\limits_{M \in {M_{\kappa}}^{\neg e}} w(M)    & + & \sum\limits_{M \in M_{\kappa}^e} w(M) \\
                & = & w(\kappa \mbox{ in } G_k^w)
        \end{array}$
    \end{center}

    So, since the weight of each feasible vertex colouring in $G_k^w$ remains unchanged in ${G'_{k'}}^{w'}$, the monochromatic feasible vertex colourings have still a weight of 1, and the non-monochromatic feasible vertex colourings have still a weight of 0.
    So, ${G'_{k'}}^{w'}$ is still perfectly monochromatic and $\Tilde{c}(G', k', w') = \Tilde{c}(G, k, w)$.
    Now, we can rename $G'$ as $G$ and repeat the whole procedure while $G_k^w$ has still edges with a weight different from $\{-1, 1\}$.
    The resulting graph has only edges that have signed unitary weights and has the same weighted chromatic index as the initial graph.
    So if any upper bound can be found on the weighted matching index of the final graph with signed unitary weights, it will still be valid for the weighted matching index of the initial one, with integer weights.
\end{proof}

The implication of this lemma is that there are actually $2$ ways to reason about the Krenn's conjecture when we are interested in integer weights.
The first way is to consider only non-redundant graphs (as defined in definition~\ref{def:non_redundant_induced_subgraph}) and to try to find a bound on their weighted matching index.
And the second way is to consider redundant graphs in which each edge has a weight included in $\{-1, 1\}$.
Every result discovered in the second approach can be translated to the first approach, and vice versa.


\section{Constraints' relaxation}
\label{sec:constraints-relaxation}

As it was explained in the introduction, a simplified version of the conjecture was already proven thanks to Bogdanov.\cite{bogdanov}
This version, presented in lemma~\ref{lem:real_pos_weights}, is only valid when all the weights of a perfectly monochromatic graph $G_k^w$ are positive.
In this section, our main goal will be to relax these constraints.

\subsection{Allowing one negative edge}
\label{subsec:one_negative_edge}

Since the conjecture is proven to be true when all the weights are positive, it is natural to ask ourselves how the proof would be affected if this constraint was relaxed.
The most simple case is the one where one edge is allowed to have a negative weight.
The goal of this section is to answer that question.
Let's begin by remarking the $2$ following observations.

\begin{observation}[Existence of a Hamiltonian cycle]
    \label{obs:2_positive_classes_ham_cycle}
    Let $G_k^w$ be a perfectly monochromatic graph that respects the following properties.
    \begin{itemize}
        \item $G_K^w$ is a simple graph, in the sense it has no multi-edges.
        \item $G_K^w$ has no bicoloured edges.
        \item $G_k^w$ has a weighted matching index $\Tilde{c}(G, k, w) \geq 3$.
        \item It has at least $2$ colour classes, denoted $r$, and $g$, such that all the edges coloured $r$ or $g$ have a real, positive weight.
    \end{itemize}

    Let $M_r$ and $M_g$ be $2$ monochromatic perfect matchings of $G_k^w$ coloured $r$ and $g$ respectively.
    Then, the union of $M_r$ and $M_g$ forms a Hamiltonian cycle of even length.
\end{observation}

\begin{proof}
    Since $M_r$ and $M_g$ are disjoint, they form a disjoint union of $\mathcal{C}$ cycles of even length.
    If $\mathcal{C} \geq 2$, we will denote by $C_i$ the $i^{th}$ cycle.
    Then, we can build the following non-monochromatic perfect matching :
    \begin{center}
        $N = (C_1 \cap M_r) \cup \left(\bigcup\limits_{i=2}^{\mathcal{C}} C_i \cap M_g\right)$
    \end{center}

    The construction of $N$ is highlighted in figure~\ref{fig:demo_unique_neg_ham}.

    \begin{figure}[H]
        \ctikzfig{figures/new_results/2_pos_classes/2_pos_classes_ham_cycle}
        \caption{In this example, the non-monochromatic perfect matching $N$ (represented by thick edges) is constructed from a red perfect matching and a blue one. The induced vertex colouring is also visible.}
        \label{fig:demo_unique_neg_ham}
    \end{figure}

    Since $M_r$ and $M_g$ include no negatively weighted edge, $w(N) > 0$.
    But, by definition~\ref{def:perfectly_monochromatic_graph} of a perfectly monochromatic graph, and using the notations introduced in definitions~\ref{def:matching_weight} and~\ref{def:feasible_vertex_colouring},

    \begin{center}
        $w(\kappa(N)) = \sum\limits_{N_i \in \mathcal{M}_{\kappa(N)}} N_i = 0$
    \end{center}

    Therefore, we know that $\exists N'$ such that $\kappa(N') = \kappa(N)$ and $w(N') < 0$.
    This is impossible, because the only way for $N'$ to have a negative weight is to include at least one negative edge.
    And negative edges don't exist in colour $r$ nor $g$.
    We can conclude that $\mathcal{C} = 1$, which means that $M_r$ and $M_g$ form a Hamiltonian cycle.
\end{proof}

\begin{observation}[Parity of crossing edges]
    \label{obs:2_positive_classes_parity_crossing_edge}
    Let $G_k^w$ be a perfectly monochromatic graph that respects the following properties.
    \begin{itemize}
        \item $G_K^w$ is a simple graph, in the sense it has no multi-edges.
        \item $G_K^w$ has no bicoloured edges.
        \item $G_k^w$ has a weighted matching index $\Tilde{c}(G, k, w) \geq 3$.
        \item It has at least $2$ colour classes, denoted $r$, and $g$, such that all the edges coloured $r$ or $g$ have a real, positive weight.
    \end{itemize}
    Let $M_r, M_g$ be $2$ monochromatic perfect matchings of colours $r$ and $g$ respectively.
    Let $H = (v_1, \dots, v_n)$ be the Hamiltonian cycle of $G_k^w$ formed by $M_r$ and $M_g$.
    Let $e = (v_i, v_j)$ be an edge whose colour is $b$.
    Then, $j-i$ is even.
\end{observation}

\begin{proof}
    Let's assume by contradiction that $j-i$ is odd.
    Without loss of generality, we can assume that the colour of $(v_i, v_{i+1})$ is $r$ (otherwise, inverse colours $r$ and $g$ in the following arguments).
    Then the following non-monochromatic perfect matching can be built.

    \begin{center}
        $N = e \cup (M_g \cap H_{i+1, j-1}) \cup (M_r \cap H_{j+1, i-1})$
    \end{center}

    The construction of the non-monochromatic perfect matching $N$ is shown in figure~\ref{fig:2_pos_classes_odd_crossings}.

    \begin{figure}[H]   % TODO : check the colours of this figure
        \ctikzfig{figures/new_results/unique_neg/unique_neg_odd_crossings}
        \caption{Construction of $N$ from $M_r$, $M_g$ and $e$ if $j-i$ is odd.}
        \label{fig:2_pos_classes_odd_crossings}
    \end{figure}

    \begin{itemize}
        \item If $w(e) > 0$, then $w(N) > 0$ by definition~\ref{def:matching_weight}.
        But $w(\kappa(N)) = \sum\limits_{N_i \in \mathcal{M}_{\kappa(N)}} N_i = 0$ by definitions~\ref{def:weighted_matching_index} and~\ref{def:perfectly_monochromatic_graph} (using a notation introduced in definition \ref{def:induced_vertex_colouring}).
        Therefore, $\exists$ another non-monochromatic perfect matching $N'$ such that $\kappa(N') = \kappa(N)$ and $w(N') < 0$.
        To satisfy this constraint, $e \in N'$.
        But $e$ is the only edge that has a colour different from $r$ nor $g$ in $N'$, which means that the sign of $w(e)$ determines the sign of $w(N')$ by definition~\ref{def:matching_weight}.
        Therefore, $w(N')$ can not be negative.
        This is a contradiction.

        \item If $w(e) < 0$, then $w(N) < 0$ by definition~\ref{def:matching_weight}.
        But $w(\kappa(N)) = \sum\limits_{N_i \in \mathcal{M}_{\kappa(N)}} N_i = 0$ by definitions~\ref{def:vertex_colouring_weight} and~\ref{def:perfectly_monochromatic_graph} (using a notation introduced in definition \ref{def:induced_vertex_colouring}).
        Therefore, $\exists$ another non-monochromatic perfect matching $N'$ such that $\kappa(N') = \kappa(N)$ and $w(N') > 0$.
        To satisfy this constraint, $e \in N'$.
        But $e$ is the only edge that has a colour different from $r$ nor $g$ in $N'$, which means that the sign of $w(e)$ determines the sign of $w(N')$ by definition~\ref{def:matching_weight}.
        Therefore, $w(N')$ can not be positive.
        This is a contradiction.

    \end{itemize}
\end{proof}

These observations may seem obscure at first, but they are necessary to prove the following lemma.

\begin{lemma}[One negative edge allowed]
    \label{lem:one_neg_edge}
    Let $G_k^w$ be a perfectly monochromatic graph that respects the following properties.
    \begin{itemize}
        \item $G_K^w$ is a simple graph, in the sense it has no multi-edges.
        \item $G_K^w$ has no bicoloured edges.
        \item $G_k^w$ has a weighted matching index $\Tilde{c}(G, k, w) \geq 3$.
        \item $\forall e \in E(G_k^w)$, $w(e) \in \mathbb{R}$.
        Also, $E(G_k^w)$ has at most one edge that has a negative weight.
    \end{itemize}
    Then, if $G_k^w$ is not isomorphic to $K_4$, $\Tilde{c}(G, k, w) \leq 2$.
\end{lemma}

The sketch of the proof of lemma~\ref{lem:one_neg_edge} goes as follow.
Using observations~\ref{obs:2_positive_classes_ham_cycle} and~\ref{obs:2_positive_classes_parity_crossing_edge}, we build a Hamiltonian cycle that has two colours and build non-monochromatic perfect matchings in it that use a negative edge $e$.
We then show that they create a disbalance in the weight of their feasible vertex colouring that can not be counterbalanced with another perfect matching.
Let's dive into it.

\begin{proof}[Proof of lemma \ref{lem:one_neg_edge}]

    Let assume by contradiction that the weighted matching index of $G_k^w$ $\Tilde{c}(G, k, w) \geq 3$.

    \begin{enumerate}
        \item[]

        \item If $G_k^w$ has only positive weights, then we're done — this case is already solved by Bogdanov in~\cite{bogdanov} and was presented in the lemma~\ref{lem:real_pos_weights} of this master thesis.

        \item If $G_k^w$ has exactly one negative weight: let $M_r$, $M_g$ and $M_b$ be three distinct monochromatic perfect matchings of $G_k^w$ that have colour $r$, $g$ and $b$ respectively.
        They exist by definition~\ref{def:weighted_matching_index} of the weighted matching index.
        Let $e^-$ be the only negatively weighted edge of $G_k^w$.
        Without loss of generality, I will say that the colour of $e^-$ is $b$.
        From observation~\ref{obs:2_positive_classes_ham_cycle}, $M_r$ and $M_g$ form a Hamiltonian cycle $H = (v_1, v_2, \dots, v_n)$ of even length.
        In this new vertex ordering, $e^- = (v_i, v_j)$.
        We know from observation~\ref{obs:2_positive_classes_parity_crossing_edge} that $j-i$ is even.
        Without loss of generality, I can assume that the color of $(v_i, v_{i + 1})$ is $r$ (otherwise we exchange colours $r$ and $g$ in the following reasoning).
        Let $e = (v_{i + 1}, v_k) \in M_b$ (we are certain of the existence of $e$ because $v_{i+1}$ must be covered by $M_b$). Now, three situations can occur.

        \begin{enumerate}
            \item If $v_k = v_j$: this situation is impossible from observation~\ref{obs:2_positive_classes_parity_crossing_edge}, because $k - (i + 1)$ is odd.    % TODO : add a figure to illustrate this case

            \item If $v_k$ appears in an edge from the arc $H_{j+1, i-1}$ (using the notation introduced in definition~\ref{def:arc} of an arc), then we can form a non-monochromatic perfect matching as follows.

                \begin{center}
                    $\begin{array}{r c l}
                        N & = & e^-                                 \\
                          &   & \cup e                              \\
                          &   & \cup (H_{j+1, k-1} \cap M_g)        \\
                          &   & \cup (H_{i + 2, j - 1} \cap M_r)    \\
                          &   & \cup (H_{k+1, i-1} \cap M_r)
                    \end{array}$
                \end{center}

                This works because we are sure that $k - (i + 1)$ is even from observation~\ref{obs:2_positive_classes_parity_crossing_edge}.
                Since $e^- \in N$, $w(N) < 0$ by definition~\ref{def:matching_weight}.
                But $w(\kappa(N)) = \sum\limits_{N_i \in \mathcal{M}_{\kappa(N)}} N_i = 0$ by definitions~\ref{def:vertex_colouring_weight} and~\ref{def:perfectly_monochromatic_graph}.
                Therefore, $\exists$ a non-monochromatic perfect matching $N'$ such that $\kappa(N') = \kappa(N)$ and $w(N') > 0$.
                This is possible only if $e^- \notin N'$.
                The only vertices that get colour $b$ in $\kappa(N) = \kappa(N')$ are $v_i, v_{i+1}, v_j$ and $v_k$.
                The only way to match these vertices in $N'$ with $b$-coloured edges without using $e^-$ is that $\exists e' = (v_i, v_k)$ and $e'' = (v_{i+1}, v_j)$ that have colour $b$ and that are included in $N'$.
                But this is impossible, because $k-i$ and $j-(i+1)$ are odd numbers, which is forbidden by observation~\ref{obs:2_positive_classes_parity_crossing_edge}.
                This reasoning is summarized in figure~\ref{fig:unique_neg_2_2}.

                \begin{figure}[H]
                    \ctikzfig{figures/new_results/unique_neg/unique_neg_2_2}
                    \caption{Illustration of the reasoning in case 2.2. In this figure, $N$ is represented by thick edges, and the induced vertex colouring from $N$ is visible. The edges $e'$ and $e''$ that should be in $N'$ are also represented. It is clear in this figure that $e'$ and $e''$ can't exist using the parity argument.}
                    \label{fig:unique_neg_2_2}
                \end{figure}

            \item If $v_k$ appears in an edge from $H_{i+1, j-1}$, then it is possible to find an edge of $M_b$, called $e' = (v_l, v_m)$, that has both endpoints in $H_{i+1, k}(H)$, and such that the edge $e'' = (v_{l+1}, v_n) \in M_b$ has its $v_n$-endpoint in $H_{m+1, l-1}(H)$.
            This is true because it is known from observation~\ref{obs:2_positive_classes_parity_crossing_edge} that $m-l$ is even.
            Actually, $e'$ might be equal to $e$.
            Assuming without loss of generality that the colour of $(v_l, v_{l+1})$ is $r$, we find a new non-monochromatic perfect matching.

                \begin{center}
                    $\begin{array}{r c l}
                        N & = & e'                              \\
                          &   & \cup e''                        \\
                          &   & \cup (H_{l+2, m-1} \cap M_r)    \\
                          &   & \cup (H_{n+1, l-1} \cap M_r)    \\
                          &   & \cup (H_{m+1, n-1} \cap M_g)
                    \end{array}$
                \end{center}

                The construction of $N$ is illustrated in figure~\ref{fig:unique_neg_2_3}.

                \begin{figure}[H]
                    \ctikzfig{figures/new_results/unique_neg/unique_neg_2_3}
                    \caption{Illustration of the reasoning in case 2.3. In this figure, $N$ is represented by thick edges, and the induced vertex colouring from $N$ is visible. It this particular case, $v_j = v_n$. Nevertheless, it is impossible for $v_i$ and $v_j$ to be both green in $\kappa(N)$.}
                    \label{fig:unique_neg_2_3}
                \end{figure}

                Since $e^- \notin N$, $w(N) > 0$ by definition~\ref{def:matching_weight}.
                But $w(\kappa(N)) = \sum\limits_{N_i \in \mathcal{M}_{\kappa(N)}} N_i = 0$ by definition~\ref{def:vertex_colouring_weight} and~\ref{def:perfectly_monochromatic_graph}.
                Therefore, $\exists$ another non-monochromatic perfect matching $N'$ such that $\kappa(N') = \kappa(N)$ and $w(N') < 0$.
                For this condition to be satisfied, $e^- \in N'$.
                But it is impossible that both $v_i$ and $v_j$ get colour $3$ in $\kappa(N) = \kappa(N')$ (only one of them maximum).
                This means that $e^-$ can't be in $N'$, which forms a contradiction.
        \end{enumerate}
    \end{enumerate}

\end{proof}

\subsection{Allowing all the classes to have arbitrary weights except 2}
\label{subsec:2-pos-classes}

The main argument of the proof of the previous analysed case was that 2 colour classes had only positive weighted edges.
This suggests that the structure of the proof might work as well if more than one single negatively weighted edge was present in the combination of the other colour classes.
Such a result would be even more powerful since it would prove the conjecture to be true whenever 2 colour classes have only positive weighted edges, no matther the weights of the other colour classes.
In this section, we will reuse the arguments from section~\ref{subsec:one_negative_edge} to verify them in the situation where multiple negative edge-weights are allowed.


\begin{lemma}[2 positive colour classes]
    \label{lem:2_positive_colour_classes_forbidden}
    Let $G_k^w$ be a perfectly monochromatic graph that respects the following properties.
    \begin{itemize}
        \item $G_K^w$ is a simple graph, in the sense it has no multi-edges.
        \item $G_K^w$ has no bicoloured edges.
        \item $\exists$ two colour classes $r$ and $g$ such that all the edges coloured $r$ or $g$ have a real, positive weight.
    \end{itemize}
    Then, if $G_k^w$ is not isomorphic to $K_4$, $\Tilde{c}(G, k, w) \leq 2$.
\end{lemma}

\begin{proof}
    By contradiction, let assume that $\Tilde{c}(G, k, w) \geq 3$.
    Let $M_r$, $M_g$ and $M_b$ be 3 distinct monochromatic perfect matchings of $G_k^w$ that have colour $r$, $g$ and $b$ respectively, and let assume that the colour classes $r$ and $g$ have only positive weighted edges.
    From observation~\ref{obs:2_positive_classes_ham_cycle}, $M_r$ and $M_g$ form a Hamiltonian cycle $H = (v_1, v_2, \dots, v_n)$ of even length.
    We know from observation~\ref{obs:2_positive_classes_parity_crossing_edge} that $\forall e = (v_i, v_j) \in M_b$, $j-i$ is even.
    Without loss of generality, let's assume that the color of $(v_i, v_{i + 1})$ is $r$ (otherwise we exchange colours $r$ and $g$ in the following arguments).
    Let $e = (v_i, v_j) \in M_b$ be a minimal crossing edge, i.e.\ be such that $e' = (v_{i+1}, v_k) \in M_b$ has its $v_k$ endpoint in $H_{j+1, i-1}$.
    It is then possible to find a non-monochromatic perfect matching as follows.

    \begin{center}
        $\begin{array}{r c l}
             N & = & e                           \\
             &   & \cup e'                       \\
             &   & \cup M_1 \cap P_{i+2, j-1}(H) \\
             &   & \cup M_1 \cap P_{k+1, i-1}(H) \\
             &   & \cup M_2 \cap P_{j+1, k-1}(H)
        \end{array}$
    \end{center}

    The construction of $N$ is visualized in figure~\ref{fig:2_pos_classes_proof}.

    \begin{figure}[H]
        \ctikzfig{figures/new_results/2_pos_classes/2_pos_classes_proof}
        \caption{Construction of $N$ from $M_1$, $M_2$, $e$ and $e'$. $N$ is represented by the thick edges. $\kappa(N)$ is also represented.}
        \label{fig:2_pos_classes_proof}
    \end{figure}

    Centering the analysis around the potential signs of $w(e)$ and $w(e')$, $2$ situations can occur.

    \begin{enumerate}
        \item if $w(e)$ and $w(e')$ have the same sign: then $w(N) > 0$ by definition~\ref{def:matching_weight}.    % TODO : add a figure to illustrate these 2 cases
        But $w(\kappa(N)) = \sum\limits_{N_i \in \mathcal{M}_{\kappa(N)}} N_i = 0$ by definitions~\ref{def:vertex_colouring_weight} and~\ref{def:perfectly_monochromatic_graph}.
        This means that $\exists N'$ such that $\kappa(N') = \kappa(N)$ and $w(N') < 0$.
        Since the sign of $N'$ is defined by the sign of its $b$-coloured edges, it can not contain $e$ and $e'$.
        This last condition is satisfied only if $\exists e'' = (v_i, v_k)$ and $e''' = (v_{i+1}, v_j)$ of colour $b$.
        But this is forbidden by observation~\ref{obs:2_positive_classes_parity_crossing_edge} because $(k - i)$ and $\left(j - (i + 1)\right)$ are odd numbers.

        \item if $w(e)$ and $w(e')$ have different signs: then $w(N) < 0$ by definition~\ref{def:matching_weight}.
        But $w(\kappa(N)) = \sum\limits_{N_i \in \mathcal{M}_{\kappa(N)}} N_i = 0$ by definitions~\ref{def:vertex_colouring_weight} and~\ref{def:perfectly_monochromatic_graph}.
        This means that $\exists N'$ such that $\kappa(N') = \kappa(N)$ and $w(N') > 0$.
        Since the sign of $N'$ is defined by the sign of its $b$-coloured edges, it can not contain $e$ and $e'$.
        This last condition is satisfied only if $\exists e'' = (v_i, v_k)$ and $e''' = (v_{i+1}, v_j)$ of colour $b$.
        But this is forbidden by observation~\ref{obs:2_positive_classes_parity_crossing_edge} because $(k - i)$ and $\left(j - (i + 1)\right)$ are odd numbers.
        This ends the proof of lemma~\ref{lem:2_positive_colour_classes_forbidden}.
    \end{enumerate}
\end{proof}
