\chapter{New theoretical contributions}
\label{ch:new-contributions}

There are two intuitive ways to tackle the problem of the Krenn's conjecture.
The first way, done in this chapter, is to use a theoretical approach to find interesting properties about perfectly monochromatic graphs.
In this approach, I prove the Krenn's conjecture in a very restrained subcase in subsection~\ref{subsec:one_negative_edge}.
Then, I show how to relax the constraints of this analysed subcase in the subsection~\ref{subsec:2-pos-classes}.
Also, I present two different ways of thinking about the problems by showing an equivalency between two cases in the subsection~\ref{sec:problem-reduction}.\\

The second approach is a computational approach, and is presented in chapter~\ref{ch:computational-approach}.
The goal of this second approach will be to generate a big number of random perfectly monochromatic experiment graphs, and extract conclusions from the data.
To do so, I will present a new tool I developed, called EGPI, that aims to perform such experiments.


\section{Problem reduction}
\label{sec:problem-reduction}

Let $\beta \in \mathbb{N}_0$ be a strictly positive integer.
Having $\beta$, we can formulate the two following conjectures.

\begin{conjecture}[Weighted matching index bounded by $\beta$ if signum weighting]
    \label{con:c-bounded-by-beta-binary}
    Let $G_\mu^w$ be a perfectly monochromatic graph such that $w: E(G) \rightarrow \{-1, 1\}$.
    Then, $\Tilde{c}(G, \mu, w) \leq \beta$.
\end{conjecture}

\begin{conjecture}[Weighted matching index bounded by $\beta$ if integer weights]
    \label{con:c-bounded-by-beta-integer}
    Let $G_\mu^w$ be a perfectly monochromatic graphs such that $w: E(G) \rightarrow \mathbb{Z}$.
    Then, $\Tilde{c}(G, \mu, w) \leq \beta$.
\end{conjecture}

Since the conjecture~\ref{con:c-bounded-by-beta-binary} is a particular case of the conjecture~\ref{con:c-bounded-by-beta-integer}, the conjecture~\ref{con:c-bounded-by-beta-binary} seems to be easier to prove.
However, the following lemma holds.

\begin{lemma}[Conjectures \ref{con:c-bounded-by-beta-binary} and \ref{con:c-bounded-by-beta-integer} are equivalent]
    \label{lem:pm_graphs_with_integer_weights}
    Let $\beta \in \mathbb{N}_0$ be a strictly positive integer.
    The conjecture~\ref{con:c-bounded-by-beta-binary} is true for $\beta$ if and only if the conjecture~\ref{con:c-bounded-by-beta-integer} is true for $\beta$.
    In other words, the two conjectures are equivalent.
\end{lemma}

\begin{proof}[Proof of Lemma \ref{lem:pm_graphs_with_integer_weights}]
    Let $G_\mu^w$ be a perfectly monochromatic graph such that $w: E(G) \rightarrow \mathbb{Z}$.
    We will show that we can build a perfectly monochromatic graph that has only weights included in $\{-1, 1\}$ and that has the same weighted matching index as $G_\mu^w$. \\

    First, we choose an edge in $G_\mu^w$ that has a weight different from $1$ or $-1$ (if such an edge doe not exist, we are done).
    We replace this edge by $|w(e)|$ parallel edges of the same colour, and that have all a weight of $\frac{w(e)}{|w(e)|}$ (1 if $w(e) > 0$, -1 if $w(e) < 0$).
    We say that each of these new edges is \textit{derived} from $e$.
    Remember this can be done because experiment graphs allow multi-edges by definition~\ref{def:experiment_graph}.
    This creates a new graph ${G'_{\mu'}}^{w'}$.
    That process is illustrated in figure~\ref{fig:demo_integers}.\\

    \begin{figure}[H]
        \ctikzfig{figures/new_results/problem_reduction/demo_integers}
        \caption{Illustration of the transformation from an experiment graph with integer weights to an experiment graph with weights included in $\{-1, 1\}$.}
        \label{fig:demo_integers}
    \end{figure}

    Let $\kappa$ be a feasible vertex colouring of $G_\mu^w$.
    To compute the weight of $\kappa$ in $G_\mu^w$, we will denote by
    \begin{center}
        $\left\{
            \begin{array}{ll}
                M_{\kappa}          & \mbox{the set of perfect matchings of } G_\mu^w \mbox{ that induce the vertex colouring } \kappa \\
                M_{\kappa}^e        & \mbox{the set of perfect matchings of } G_\mu^w \mbox{ that induce the vertex colouring } \kappa \mbox{ and contain } e \\
                M_{\kappa}^{\neg e} & \mbox{the set of perfect matchings of } G_\mu^w \mbox{ that induce the vertex colouring } \kappa \mbox{ and do not contain } e
            \end{array}
        \right.$
    \end{center}

    The weight of $\kappa$ in $G_\mu^w$ is

    \begin{center}
        $\begin{array}{lclcl}
            w(\kappa \mbox{ in } G_\mu^w)
                & = & \sum\limits_{M \in M_{\kappa}} w(M) \\
                & = & \sum\limits_{M \in M_{\kappa}^{\neg e}} w(M) & + & \sum\limits_{M \in M_{\kappa}^{e}} w(M)
        \end{array}$
    \end{center}

    The next step, which is the heart of the proof, consists of computing the weight of the vertex colouring $\kappa$ in ${G'_{\mu'}}^{w'}$.
    We will denote by

    \begin{center}
        $\left\{
            \begin{array}{ll}
                M'_{\kappa}            & \mbox{the set of perfect matchings of } {G'_{\mu'}}^{w'} \mbox{ that induce the vertex colouring } \kappa \\
                {M'_{\kappa}}^e        & \mbox{the set of perfect matchings of } {G'_{\mu'}}^{w'} \mbox{ that induce the vertex colouring } \kappa \\
                                   & \mbox{and contain an edge that was derived from } e \\
                {M'_{\kappa}}^{\neg e} & \mbox{the set of perfect matchings of } {G'_{\mu'}}^{w'} \mbox{ that induce the vertex colouring } \kappa \\
                                   & \mbox{and does not contain an edge that was derived from } e
            \end{array}
        \right.$
    \end{center}

    In addition to these concepts, for every perfect matching $M \in M_{\kappa}^e$, let $\mathcal{M}'(M)$ be the set of corresponding perfect matchings $M'$ in ${M'_{\kappa}}^e$.
    In other words, $\mathcal{M}'(M)$ denotes the set of perfect matchings that are the same as $M$ on every edge except $e$, and that contain one of the edges that were added when $e$ was removed.
    It follows that, for every $M \in M_{\kappa}^e$, $|\mathcal{M}'(M)| = w(e)$.
    Also, given $M \in M_{\kappa}^e$, for each perfect matching $M' \in \mathcal{M}'(M)$, $w(M') = \frac{w(M)}{|w(e)|}$.
    Finally, we notice the following relations between the different sets we defined :

    \begin{center}
        $\left\{
            \begin{array}{lcl}
                {M'_{\kappa}}^{\neg e} & = & M_{\kappa}^{\neg e} \\
                {M'_{\kappa}}^e        & = & \bigcup\limits_{M \in M_{\kappa}^e} \mathcal{M}'(M)
            \end{array}
        \right.$
    \end{center}

    Having all these observations in mind, we can now compute the weight of $\kappa$ in ${G'_{\mu'}}^{w'}$.

    \begin{center}
        $\begin{array}{lclcl}
            w(\kappa \mbox{ in } {G'_{\mu'}}^{w'})
                & = & \sum\limits_{M' \in M'_{\kappa}} w(M') \\
                & = & \sum\limits_{M' \in {M'_{\kappa}}^{\neg e}} w(M') & + & \sum\limits_{M' \in {M'_{\kappa}}^{e}} w(M') \\
                & = & \sum\limits_{M \in {M_{\kappa}}^{\neg e}} w(M)    & + & \sum\limits_{M \in M_{\kappa}^e} \left( \sum\limits_{M' \in \mathcal{M}'(M)} w(M') \right) \\
                & = & \sum\limits_{M \in {M_{\kappa}}^{\neg e}} w(M)    & + & \sum\limits_{M \in M_{\kappa}^e} \left( \sum\limits_{M' \in \mathcal{M}'(M)} \frac{w(M)}{|w(e)|} \right) \\
                & = & \sum\limits_{M \in {M_{\kappa}}^{\neg e}} w(M)    & + & \sum\limits_{M \in M_{\kappa}^e} \left( |w(e)| \frac{w(M)}{|w(e)|} \right) \\
                & = & \sum\limits_{M \in {M_{\kappa}}^{\neg e}} w(M)    & + & \sum\limits_{M \in M_{\kappa}^e} w(M) \\
                & = & w(\kappa \mbox{ in } G_\mu^w)
        \end{array}$
    \end{center}

    So, since the weight of each feasible vertex colouring in $G_\mu^w$ remains unchanged in ${G'_{\mu'}}^{w'}$, the monochromatic feasible vertex colourings have still a weight of 1, and the non-monochromatic feasible vertex colourings have still a weight of 0.
    So, ${G'_{\mu'}}^{w'}$ is still perfectly monochromatic and $\Tilde{c}(G', \mu', w') = \Tilde{c}(G, \mu, w)$.
    Now, we can rename $G'$ as $G$ and repeat the whole procedure while $G_\mu^w$ has still edges with a weight different from $\{-1, 1\}$.
    The resulting graph has only edges that have signed unitary weights and has the same weighted chromatic index as the initial graph.
    So if any upper bound can be found on the weighted matching index of the final graph with signed unitary weights, it will still be valid for the weighted matching index of the initial one, with integer weights.
\end{proof}

The implication of Lemma~\ref{lem:pm_graphs_with_integer_weights} is that there are actually two ways to reason about the Krenn's conjecture when we are interested in integer weights.
The first way is to consider only non-redundant graphs (as defined in definition~\ref{def:redundant-experiment-graph}) and to try to find a bound on their weighted matching index.
And the second way is to consider redundant graphs in which each edge has a weight included in $\{-1, 1\}$.
Every result discovered in the second approach can be translated to the first approach, and vice versa.


\section{Constraints' relaxation}
\label{sec:constraints-relaxation}

As it was explained in the introduction, a simplified version of the conjecture was already proven thanks to Bogdanov.\cite{bogdanov}
This version, presented in Lemma~\ref{lem:real_pos_weights}, is only valid when all the weights of a perfectly monochromatic graph $G_\mu^w$ are positive.
In this section, our main goal will be to relax these constraints.

\subsection{Allowing one negative edge}
\label{subsec:one_negative_edge}

Since the conjecture is proven to be true when all the weights are positive, it is natural to ask ourselves how the proof would be affected if this constraint is relaxed.
The most simple case is the one where one edge is allowed to have a negative weight.
In this section, I show that the Krenn's conjecture is true for simple graphs in the absence of bicoloured edges and when maximum one edge has a negative weight.
Let's begin by proving the two following claims.

\begin{claim}[Existence of a Hamiltonian cycle]
    \label{clm:2_positive_classes_ham_cycle}
    Let $G_\eta^w$ be a perfectly monochromatic graph that respects the following properties.
    \begin{itemize}
        \item $G$ is a simple graph, referring to definition~\ref{def:simple_graph}.
        \item $\eta$ is a pure edge colouring, referring to definition~\ref{def:edge-coloured-graph}.
        \item $G_\eta^w$ has a weighted matching index $\Tilde{c}(G, \eta, w) \geq 3$.
        \item $\exists$ two colours $r, g \in \left(\eta(E(G))\right)^2, r \neq g$, such that all the edges coloured $r$ or $g$ in $G_\eta^w$ have a real, positive weight.
    \end{itemize}

    Let $M_r$ and $M_g$ be $2$ monochromatic perfect matchings of $G_\eta^w$ coloured $r$ and $g$ respectively.
    Then, the union of $M_r$ and $M_g$ forms a Hamiltonian cycle of even length.
\end{claim}

\begin{proof}[Proof of Claim \ref{clm:2_positive_classes_ham_cycle}]
    Since $M_r$ and $M_g$ are disjoint, they form a disjoint union $\mathcal{C}$ of cycles of even length (this was formally showed in Claim~\ref{clm:even_cycles}).
    If $\left| \mathcal{C} \right| \geq 2$, we denote by $C_i$ the $i^{th}$ cycle.
    Then, we can build the following non-monochromatic perfect matching :
    \begin{center}
        $N = (C_1 \cap M_r) \cup \left(\bigcup\limits_{i=2}^{\left| \mathcal{C} \right|} C_i \cap M_g\right)$
    \end{center}

    The construction of $N$ is highlighted in figure~\ref{fig:demo_unique_neg_ham}.

    \begin{figure}[H]
        \ctikzfig{figures/new_results/2_pos_classes/2_pos_classes_ham_cycle}
        \caption{In this example, the non-monochromatic perfect matching $N$ (represented by thick edges) is constructed from a red perfect matching and a blue one. The induced vertex colouring is also visible.}
        \label{fig:demo_unique_neg_ham}
    \end{figure}

    Since $M_r$ and $M_g$ include no negatively weighted edge, $w(N) = \prod\limits_{e \in N} w(e) > 0$.
    But, by definition~\ref{def:perfectly_monochromatic_graph} of a perfectly monochromatic graph, and using the notations introduced in definitions~\ref{def:matching_weight} and~\ref{def:feasible_vertex_colouring},

    \begin{center}
        $w(\kappa(N)) = \sum\limits_{N_i \in \mathcal{M}_{\kappa(N)}} N_i = 0$
    \end{center}

    Therefore, we know that $\exists N' \in \mathcal{M}_{\kappa(N)}$ such that $w(N') < 0$.
    This is impossible, because the only way for $N'$ to have a negative weight is to include at least one negative edge.
    And negative edges don't exist in colours $r$ or $g$.
    We can conclude that $\left| \mathcal{C} \right| = 1$, which means that the union of $M_r$ and $M_g$ forms a Hamiltonian cycle.
\end{proof}

\begin{claim}[Parity of crossing edges]
    \label{clm:2_positive_classes_parity_crossing_edge}
    Let $G_\eta^w$ be a perfectly monochromatic graph that respects the following properties.
    \begin{itemize}
        \item $G$ is a simple graph, referring to definition~\ref{def:simple_graph}.
        \item $\eta$ is a pure edge colouring, referring to definition~\ref{def:edge-coloured-graph}.
        \item $G_\eta^w$ has a weighted matching index $\Tilde{c}(G, \eta, w) \geq 3$.
        \item $\exists$ two colours $r, g \in \left(\eta(E(G))\right)^2, r \neq g$, such that all the edges coloured $r$ or $g$ in $G_\eta^w$ have a real, positive weight.
    \end{itemize}
    Let $M_r, M_g$ be $2$ monochromatic perfect matchings of different colours $r$ and $g$ respectively.
    Let $H = (v_1, \dots, v_n)$ be the Hamiltonian cycle of $G_\eta^w$ formed by $M_r$ and $M_g$.
    Let $e = (v_i, v_j) \in E(G)$ be an edge whose colour is $b \notin \{r, g\}$.
    Then, $j-i$ is even.
\end{claim}

\begin{proof}[Proof of Claim \ref{clm:2_positive_classes_parity_crossing_edge}]
    Let's assume by contradiction that $j-i$ is odd.
    Without loss of generality, we can assume that the colour of $(v_i, v_{i+1})$ is $r$ (otherwise, inverse colours $r$ and $g$ in the following arguments).
    Then the following non-monochromatic perfect matching can be built.

    \begin{center}
        $N = e \cup (M_g \cap H_{i+1, j-1}) \cup (M_r \cap H_{j+1, i-1})$
    \end{center}

    The construction of the non-monochromatic perfect matching $N$ is shown in figure~\ref{fig:2_pos_classes_odd_crossings}.

    \begin{figure}[H]
        \ctikzfig{figures/new_results/2_pos_classes/2_pos_classes_odd_crossings}
        \caption{Construction of $N$ from $M_r$, $M_g$ and $e$ if $j-i$ is odd.
            On this figure, $N$ is represented by thick edges.
            The induced vertex colouring $\kappa(N)$ is also visible.}
        \label{fig:2_pos_classes_odd_crossings}
    \end{figure}

    \begin{itemize}
        \item If $w(e) > 0$, then, since $e$ was the only edge that had a colour different from $r$ or $g$ in $N$,
        \begin{center}
            $w(N) = \prod\limits_{e' \in N} w(e') > 0$
        \end{center}
        by definition~\ref{def:matching_weight}.
        But $w(\kappa(N)) = \sum\limits_{N_i \in \mathcal{M}_{\kappa(N)}} N_i = 0$ by definitions~\ref{def:weighted_matching_index} and~\ref{def:perfectly_monochromatic_graph} (using a notation introduced in definition \ref{def:induced_vertex_colouring}).
        Therefore, $\exists$ another non-monochromatic perfect matching $N' \in \mathcal{M}_\kappa(N)$ (using notations introduced in~\ref{def:feasible_vertex_colouring}) such that $w(N') < 0$.
        To satisfy this constraint, $e \in N'$.
        But $e$ is the only edge that has a colour different from $r$ or $g$ in $N'$, which means that the sign of $w(e)$ determines the sign of $w(N')$ by definition~\ref{def:matching_weight}.
        Therefore, $w(N')$ can not be negative.
        This is a contradiction.

        \item If $w(e) < 0$, then, since $e$ was the only edge that had a colour different from $r$ or $g$ in $N$,
        \begin{center}
            $w(N) = \prod\limits_{e' \in N} w(e') = w(e) \cdot \prod\limits_{e' \in N \setminus \{e\}} w(e') < 0$
        \end{center}
        by definition~\ref{def:matching_weight}.
        But $w(\kappa(N)) = \sum\limits_{N_i \in \mathcal{M}_{\kappa(N)}} N_i = 0$ by definitions~\ref{def:vertex_colouring_weight} and~\ref{def:perfectly_monochromatic_graph} (using a notation introduced in definition \ref{def:induced_vertex_colouring}).
        Therefore, $\exists$ another non-monochromatic perfect matching $N' \in \mathcal{M}_\kappa(N)$ such that $w(N') > 0$.
        To satisfy this constraint, $e \in N'$.
        But $e$ is the only edge that has a colour different from $r$ or $g$ in $N'$, which means that the sign of $w(e)$ determines the sign of $w(N')$ by definition~\ref{def:matching_weight}.
        Therefore, $w(N')$ can not be positive.
        This is a contradiction.

    \end{itemize}
\end{proof}

These observations may seem obscure at first, but they are necessary to prove the following lemma.

\begin{lemma}[One negative edge allowed]
    \label{lem:one_neg_edge}
    Let $G_\eta^w$ be a perfectly monochromatic graph that respects the following properties.
    \begin{itemize}
        \item $G$ is a simple graph, referring to definition~\ref{def:simple_graph}.
        \item $\eta$ is a pure edge colouring, referring to definition~\ref{def:edge-coloured-graph}.
        \item $\forall e \in E(G_\eta^w)$, $w(e) \in \mathbb{R}$.
        Also, $E(G_\eta^w)$ has at most one edge that has a negative weight.
    \end{itemize}
    Then, if $G_\eta^w$ is not isomorphic to $K_4$, $\Tilde{c}(G, \eta, w) \leq 2$.
\end{lemma}

The sketch of the proof of Lemma~\ref{lem:one_neg_edge} goes as follows.
Using observations~\ref{clm:2_positive_classes_ham_cycle} and~\ref{clm:2_positive_classes_parity_crossing_edge}, we find a Hamiltonian cycle formed of $2$ distinct monochromatic perfect matchings.
Then, we build non-monochromatic perfect matchings in it using edges from a third monochromatic perfect matching.
At last, we show that they create a disbalance in the weight of their feasible vertex colouring that can not be counterbalanced with another perfect matching.
This is done by analysing every possible location of the only negatively weighted edge allowed.
Let's dive into it.

\begin{proof}[Proof of Lemma \ref{lem:one_neg_edge}]

    Let's assume by contradiction that the weighted matching index of $G_\eta^w$ is $\Tilde{c}(G, \eta, w) \geq 3$.

    \begin{enumerate}
        \item[]

        \item If $G_\eta^w$ has only positive weights, then we're done — this case is already solved by Bogdanov in~\cite{bogdanov} and was presented in the Lemma~\ref{lem:real_pos_weights} of this master thesis.

        \item If $G_\eta^w$ has exactly one negative weight: let $M_r$, $M_g$ and $M_b$ be three distinct monochromatic perfect matchings of $G_\eta^w$ that have colours $r$, $g$ and $b$ respectively.
        They exist by definition~\ref{def:weighted_matching_index} of the weighted matching index.
        Let $e^-$ be the only negatively weighted edge of $G_\eta^w$.
        Without loss of generality, I will say that the colour of $e^-$ is $b$.
        From Claim~\ref{clm:2_positive_classes_ham_cycle}, $M_r$ and $M_g$ form a Hamiltonian cycle $H = (v_1, v_2, \dots, v_n)$ of even length.
        Let $e = \{v_i, v_j\} \in M_b$ be a minimal cutting edge of $H$, which means it respects the following property.
        \begin{center}
            $\left| H_{i, j} \right| = \min\limits_{\{v_k, v_l\} \in M_3} \left| H_{k, l} \right|$
        \end{center}

        We know from Claim~\ref{clm:2_positive_classes_parity_crossing_edge} that $j-i$ is even.
        Without loss of generality, I can assume that the color of $(v_i, v_{i + 1})$ is $r$ (otherwise we exchange colours $r$ and $g$ in the following reasoning).
        Let $e' = (v_{i + 1}, v_k) \in M_b$ (we are certain of the existence of $e'$ because $v_{i+1}$ must be covered by $M_b$).
        We observe that $v_k$ must appear in a vertex of $H_{j+1, i-1}$, because otherwise we would have $\left| H_{i, j} \right| > \left| H_{j+1, k} \right|$, which contradicts the minimality of $e$.\\

        Also, $j - i$ and $k - (i + 1)$ are even numbers, by the Claim~\ref{clm:2_positive_classes_parity_crossing_edge}.
        We can now build a non-monochromatic perfect matching $N$ as follows.

        \begin{center}
            $\begin{array}{r c l}
                N & = & \{e\}                             \\
                  &   & \cup \{e'\}                       \\
                  &   & \cup (H_{j+1, k-1} \cap M_g)      \\
                  &   & \cup (H_{i + 2, j - 1} \cap M_r)  \\
                  &   & \cup (H_{k+1, i-1} \cap M_r)
            \end{array}$
        \end{center}

        The construction of $N$ is illustrated in figure~\ref{fig:one-neg-edge}.

        \begin{figure}[H]
            \ctikzfig{figures/new_results/unique_neg/one_neg_edge}
            \caption{Illustration of the construction of $N$ using the fact that $e$ is a minimal cutting edge.
                This construction works because of the parity argument proved in Claim~\ref{clm:2_positive_classes_parity_crossing_edge}.
                The induced vertex colouring $\kappa(N)$ is also visible.}
            \label{fig:one-neg-edge}
        \end{figure}

        The weight of $N$ is computed as follows.
        \begin{center}
            $\begin{array}{r c l}
            w(N) & = & \prod\limits_{e_i \in N} w(e_i) \\
                 & = & w(e) \cdot w(e') \cdot \prod\limits_{e_i \in N \setminus \{e, e'\}} w(e_i) \\
            \end{array}$
        \end{center}

        Since $e^-$ is a $b$-coloured edge, three situations can occur.
        \begin{enumerate}
            \item If $e = e^-$, then $w(N) < 0$.
                But $w(\kappa(N)) = \sum\limits_{N_i \in \mathcal{M}_{\kappa(N)}} N_i = 0$ by definitions~\ref{def:weighted_matching_index} and~\ref{def:perfectly_monochromatic_graph} (using a notation introduced in definition \ref{def:induced_vertex_colouring}).
                Therefore, $\exists N' \in \mathcal{M}_\kappa(N)$ (defined in~\ref{def:feasible_vertex_colouring}) such that $w(N') > 0$.
                To satisfy this constraint, $e^- = e \notin N'$.\\

                The only way to match $v_i$ with a $b$-coloured edge different from $e^-$ is that $\exists$ a $b$-coloured edge $e'' = (v_i, v_k) \in N'$.
                Indeed, $e''$ can't be between $v_i$ and $v_{i+1}$ because there's already an $r$-coloured edge there, and $G$ is a simple graph.
                But this is impossible, because $k-i$ is an odd number, which is forbidden by Claim~\ref{clm:2_positive_classes_parity_crossing_edge}.
                This is a contradiction.

            \item If $e' = e^-$, then $w(N) < 0$.
                But $w(\kappa(N)) = \sum\limits_{N_i \in \mathcal{M}_{\kappa(N)}} N_i = 0$ by definitions~\ref{def:weighted_matching_index} and~\ref{def:perfectly_monochromatic_graph} (using a notation introduced in definition \ref{def:induced_vertex_colouring}).
                Therefore, $\exists N' \in \mathcal{M}_\kappa(N)$ (defined in~\ref{def:feasible_vertex_colouring}) such that $w(N') > 0$.
                To satisfy this constraint, $e^- = e' \notin N'$.\\

                The only way to match $v_{i+1}$ with a $b$-coloured edge different from $e^-$ is that $\exists$ a $b$-coloured edge $e'' = (v_{i+1}, v_j) \in N'$.
                Indeed, $e''$ can't be between $v_i$ and $v_{i+1}$ because there's already an $r$-coloured edge there, and $G$ is a simple graph.
                But this is impossible, because $j - (i+1)$ is an odd number, which is forbidden by Claim~\ref{clm:2_positive_classes_parity_crossing_edge}.
                This is a contradiction.

            \item If $e^- \notin \left\{ e, e' \right\}$, then $w(N) > 0$.
                But $w(\kappa(N)) = \sum\limits_{N_i \in \mathcal{M}_{\kappa(N)}} N_i = 0$ by definitions~\ref{def:weighted_matching_index} and~\ref{def:perfectly_monochromatic_graph} (using a notation introduced in definition \ref{def:induced_vertex_colouring}).
                Therefore, $\exists N' \in \mathcal{M}_\kappa(N)$ (using notations introduced in~\ref{def:feasible_vertex_colouring}) such that $w(N') < 0$.
                To satisfy this constraint, $e^- \in N'$.\\

                Because $e^-$ is $b$-coloured, it connects $2$ vertices $\in \left\{ v_i, v_{i+1}, v_j, v_k \right\}$
                \begin{itemize}
                    \item $e^- \neq \left\{ v_i, v_{i+1} \right\}$ because there's already an $r$-coloured edge there, and $G$ is a simple graph.
                    \item Also, $e^- \neq \left\{ v_j, v_k \right\}$, since it would imply again the existence of a $b$-coloured edge between $v_i$ and $v_{i+1}$.
                    \item At last, $e^- \neq \left\{ \left\{v_i, v_j\right\}, \left\{v_{i+1}, v_k\right\} \right\}$, because it is different from $e$ and $e'$.
                \end{itemize}

                The last possibilities are that $e^- = \left\{ v_i, v_k \right\}$ or $e^- = \left\{ v_{i+1}, v_j \right\}$.
                But this is impossible, because $k - i$ and $j - (i+1)$ are odd numbers, which is forbidden by Claim~\ref{clm:2_positive_classes_parity_crossing_edge}.
                This is a contradiction, and it ends our proof.
        \end{enumerate}
    \end{enumerate}
\end{proof}


\subsection{Allowing all the classes to have arbitrary weights except 2}
\label{subsec:2-pos-classes}

The main argument of the proof of the previous analysed case was that there were two colours containing only positively weighted edges.
This suggests that the structure of the proof might work as well if more than one single negatively weighted edge was present in the combination of the other colour classes.
Such a result would be even more powerful since it would prove the conjecture to be true whenever 2 colours have only positive weighted edges, no matter the weights of the other edges.
In this section, we will reuse the arguments from section~\ref{subsec:one_negative_edge} to verify them in the situation where multiple negative edge-weights are allowed.


\begin{lemma}[2 positive colours]
    \label{lem:2_positive_colour_classes_forbidden}
    Let $G_\eta^w$ be a perfectly monochromatic graph that respects the following properties.
    \begin{itemize}
        \item $G$ is a simple graph, referring to definition~\ref{def:simple_graph}.
        \item $\eta$ is a pure edge colouring, referring to definition~\ref{def:edge-coloured-graph}.
        \item $\exists r, g \in \left(\eta(E(G))\right)^2, r \neq g$ such that all the edges coloured $r$ or $g$ in $G_\eta^w$ have a real, positive weight.
    \end{itemize}
    Then, if $G_\eta^w$ is not isomorphic to $K_4$, $\Tilde{c}(G, \eta, w) \leq 2$.
\end{lemma}

\begin{proof}[Proof of Lemma \ref{lem:2_positive_colour_classes_forbidden}]
    By contradiction, let's assume that $\Tilde{c}(G, \eta, w) \geq 3$.
    Let $M_r$, $M_g$ and $M_b$ be 3 distinct monochromatic perfect matchings of $G_\eta^w$ that have colours $r$, $g$ and $b$ respectively, and let's assume that the colours $r$ and $g$ have only positive weighted edges.
    From Claim~\ref{clm:2_positive_classes_ham_cycle}, $M_r$ and $M_g$ form a Hamiltonian cycle $H = (v_1, v_2, \dots, v_n)$ of even length.
    Let $e = \{v_i, v_j\} \in M_b$ be a minimal cutting edge of $H$, which means it respects the following property.
    \begin{center}
        $\left| H_{i, j} \right| = \min\limits_{\{v_k, v_l\} \in M_3} \left| H_{k, l} \right|$
    \end{center}
    Let $e' = (v_{i + 1}, v_k) \in M_b$ (we are certain of the existence of $e'$ because $v_{i+1}$ must be covered by $M_b$).
    We know from Claim~\ref{clm:2_positive_classes_parity_crossing_edge} that $j-i$ and $k - (i + 1)$ are even numbers.
    We observe that $v_k$ must appear in a vertex of $H_{j+1, i-1}$, because otherwise we would have $\left| H_{i, j} \right| > \left| H_{j+1, k} \right|$, which contradicts the minimality of $e$.\\

    Without loss of generality, let's assume that the color of $\{v_i, v_{i + 1}\}$ is $r$ (otherwise we exchange colours $r$ and $g$ in the following arguments).
    It is then possible to find a non-monochromatic perfect matching as follows.

    \begin{center}
        $\begin{array}{r c l}
             N & = & \{e\}                    \\
             &   & \cup \{e'\}                \\
             &   & \cup M_r \cap H_{i+2, j-1} \\
             &   & \cup M_r \cap H_{k+1, i-1} \\
             &   & \cup M_g \cap H_{j+1, k-1}
        \end{array}$
    \end{center}

    The construction of $N$ is visualized in figure~\ref{fig:2_pos_classes_proof}.

    \begin{figure}[H]
        \ctikzfig{figures/new_results/2_pos_classes/2_pos_classes_proof}
        \caption{Construction of $N$ from $M_r$, $M_g$, $e$ and $e'$. $N$ is represented by the thick edges.
            $\kappa(N)$ is also represented.}
        \label{fig:2_pos_classes_proof}
    \end{figure}

    By definition~\ref{def:matching_weight}, the weight of $N$ computed as follows.

    \begin{center}
        $\begin{array}{r c l}
            w(N) & = & \prod\limits_{e_i \in N} w(e_i) \\
                 & = & w(e) \cdot w(e') \cdot \prod\limits_{e_i \in N \setminus \{e, e'\}} w(e_i) \\
        \end{array}$
    \end{center}

    We notice that $N \setminus \{e, e'\}$ has only $r$ and $g$-coloured edges, which have positive weights.
    Therefore, the sign of $w(N)$ is determined by the signs of $w(e)$ and $w(e')$.

    \begin{enumerate}
        \item if $w(e)$ and $w(e')$ have the same sign: then, $w(N) > 0$.

            But $w(\kappa(N)) = \sum\limits_{N_i \in \mathcal{M}_{\kappa(N)}} N_i = 0$ by definitions~\ref{def:vertex_colouring_weight} and~\ref{def:perfectly_monochromatic_graph}.
            This means that $\exists N' \in \mathcal{M}_{\kappa(N)}$ such that $w(N') < 0$.\\

            The only $b$-coloured vertices in $\kappa(N) = \kappa(N')$ are $v_i, v_{i+1}, v_j$ and $v_k$.
            For that reason, we can not have $\{e, e'\} \in N'$ (otherwise the signs of $w(N)$ and $w(N')$ would be the same).
            This last condition is satisfied only if $\exists e'' = (v_i, v_k)$ and $e''' = (v_{i+1}, v_j)$ of colour $b$.
            But this is forbidden by Claim~\ref{clm:2_positive_classes_parity_crossing_edge} because $(k - i)$ and $\left(j - (i + 1)\right)$ are odd numbers.

        \item if $w(e)$ and $w(e')$ have different signs: then, $w(N) < 0$.

            But $w(\kappa(N)) = \sum\limits_{N_i \in \mathcal{M}_{\kappa(N)}} N_i = 0$ by definitions~\ref{def:vertex_colouring_weight} and~\ref{def:perfectly_monochromatic_graph}.
            This means that $\exists N' \in \mathcal{M}_{\kappa(N)}$ such that $w(N') > 0$.\\

            The only $b$-coloured vertices in $\kappa(N) = \kappa(N')$ are $v_i, v_{i+1}, v_j$ and $v_k$.
            For that reason, we can not have $\{e, e'\} \in N'$ (otherwise the signs of $w(N)$ and $w(N')$ would be the same).
            This last condition is satisfied only if $\exists e'' = (v_i, v_k)$ and $e''' = (v_{i+1}, v_j)$ of colour $b$.
            But this is forbidden by Claim~\ref{clm:2_positive_classes_parity_crossing_edge} because $(k - i)$ and $\left(j - (i + 1)\right)$ are odd numbers.\\

            This ends the proof of Lemma~\ref{lem:2_positive_colour_classes_forbidden}.
    \end{enumerate}
\end{proof}


\section{Other explored cases}
\label{sec:other-explored-cases}

Before finding these interesting results, I explored other subcases of the Krenn's conjecture that were less successful.
Nevertheless, I could make some interesting observations in one of them: the case of bipartite graphs.
This lead me to assemble my observations into a lemma, presented in the next subsection.

\subsection{Focus on bipartite graphs}
\label{subsec:focus-on-bipartite-graphs}

Many problems in graph theory about matchings are easier to solve when restricted to bipartite graphs.
For this reason, I spent some time trying to find a proof of the Krenn's conjecture that would be valid only for bipartite graphs.
This was unsuccessful, but I will present the main observations I made during this exploration by proving the following lemma.

\begin{lemma}[Bipartite graphs]
    \label{lem:bipartite_graphs}
    Let $G_\eta^w$ be a perfectly monochromatic graph of size $n$ that respects the following properties.
    \begin{itemize}
        \item $G$ is a simple graph, referring to definition~\ref{def:simple_graph}.
        \item $\eta$ is a pure edge colouring, referring to definition~\ref{def:edge-coloured-graph}.
        \item $G$ is bipartite.
        \item $\forall e \in E(G_\eta^w)$, $w(e) \in \{-1, 1\}$
        \item $\Tilde{c}(G, \eta, w) \geq 3$
        \item $\forall$ pair of monochromatic perfect matching $M_r$ and $M_g$ of $G_\eta^w$ of different colours, $M_r$ and $M_g$ form a Hamiltonian cycle.
    \end{itemize}
    Then, $G_\eta^w$ has at least $n + 7$ distinct perfect matchings.
\end{lemma}

\begin{proof}[Proof of Lemma \ref{lem:bipartite_graphs}]
    Let $M_r$, $M_g$ and $M_b$ be three monochromatic perfect matchings of $G_\eta^w$ that have the different colours $r$, $g$ and $b$ respectively.
    Their existence is guaranteed by definition~\ref{def:weighted_matching_index} of the weighted matching index.
    Let $H = (v_1, v_2, \dots, v_n)$ be the Hamiltonian cycle formed by $M_r$ and $M_g$.
    Let $e = (v_i, v_j) \in M_b$.
    Since $G_\eta^w$ is bipartite, $j-i$ is odd.
    We can assume without loss of generality that the colour of $(v_i, v_{i+1})$ is $r$ (otherwise, we exchange colours $r$ and $g$ in the following reasoning).
    Then, we can build a non-monochromatic perfect matching as follows.

    \begin{center}
        $\begin{array}{r c l}
            N & = & \{e\}                         \\
              &   & \cup (H_{i+1, j-1} \cap M_g)  \\
              &   & \cup (H_{j+1, i-1} \cap M_r)
        \end{array}$
    \end{center}

    Since $N$ is non-monochromatic, the weight of its induced vertex colouring $w(\kappa(N)) = \sum\limits_{N_i \in \mathcal{M}_{\kappa(N)}} N_i = 0$ by definition~\ref{def:perfectly_monochromatic_graph}.
    Also, $w(N) \in \{-1, 1\}$ because all its edges have a weight in $\{-1, 1\}$.
    Therefore, $\exists N'$ such that $\kappa(N') = \kappa(N)$ and $w(N') = -w(N)$.\\

    This reasoning can be applied to all the edges of $M_b$.
    Let
    \begin{itemize}
        \item $N_e, N'_e$ be the $2$ distinct non-monochromatic perfect matchings that can be built from $e \in M_b$.
        \item $N_{e'}, N'_{e'}$ be the $2$ distinct non-monochromatic perfect matchings that can be built from $e' \in M_b$ ($e' \neq e$).
    \end{itemize}

    We observe that $\kappa(N_e) = \kappa(N'_e) \neq \kappa(N_{e'}) = N'_{e'}$.
    Indeed, the only $b$-edge in $N_e$ and $N'_e$ is $e$, and the only $b$-edge in $N_{e'}$ and $N'_{e'}$ is $e'$.
    This means that $N_e$, $N'_e$, $N_{e'}$ and $N'_{e'}$ are all distinct.\\

    This reasoning can be applied to all pair of edges of $M_b$.
    We conclude that $G_\eta^w$ has at least $2 \cdot \frac{n}{2} = n$ distinct non-monochromatic perfect matchings.\\

    Actually, the reasoning can still go a bit further.
    Having $e \in M_b$, we notice that both $N_e$ and $N'_e$ contain $e$.
    This means that the only way for $N'_e$ to differ from $N_e$ is that they differ in the edges that have colour $r$ or $g$.
    Let's say without loss of generality that they differ (at least) in their red parts.
    Let $N'_{e}\left[r\right]$ be the set of red edges of $N'_e$.
    Then, we can find a new monochromatic red perfect matching as follows.

    \begin{center}
        $\begin{array}{r c l}
            M'_r & = & N'_{e}\left[r\right] \\
                 &   & \cup (H_{j+1, i-1} \cap M_r)
        \end{array}$
    \end{center}

    We will denote by $\kappa_r$ the red monochromatic vertex colouring.
    Let's compute the current weight of $\kappa_r$.
    \begin{center}
        $w(\kappa_r) = w(M_r) + w(M'_r) \in \{-2, 0, 2\}$
    \end{center}
    Currently, this weight can not be $1$.
    This means that $\exists M''_r$, different from $M'_r$ and $M_r$, such that $\kappa(M''_r) = \kappa_r$.\\

    This leaves us with (at least) $3$ monochromatic perfect matchings that induce the vertex colouring $\kappa_r$.
    The same reasoning is possible by forming an initial hamiltonian cycle with $M_g$ and $M_b$.
    Therefore, $\kappa_g$ or $\kappa_b$ is also induced by (at least) $3$ monochromatic perfect matchings.
    The other is induced by (at least) $1$ perfect matching because $\Tilde{c}(G, \eta, w) \geq 3$.
    By summing everything up, we find that $G_\eta^w$ has at least $n + 7$ distinct perfect matchings.

\end{proof}
