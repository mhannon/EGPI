\setcounter{secnumdepth}{-1}

\chapter{Introduction}
\label{ch:introduction}

Quantum computing has hugely developed in the last few years, with the rise of new technologies and the development of new algorithms.
Advancements made in quantum information and computation promise theoretical ways to solve problems that are currently intractable with classical computers.
% TODO: continue the introduction
% TODO: add references
% TODO: count the bipartite graphs
% TODO: write conclusion

% TODO: (ASK) add preface (or abstract)
% TODO: (ASK) manage plan of realization
% TODO: (ASK) complexity of finding perfect matchings
% TODO: (ASK) complexity of is_bipartite


Graph colourings and matchings are two major concepts of graph theory that were studied and restudied since the rising of that field.
Finding new exciting results in this domain can be challenging due to the vast amount of existing research.
But recently, a new compelling problem linking the 2 subjects was posted by the physicist Mario Krenn in~\cite{wordpress} in the context of its study of quantum physics questions.
We will refer to this problem as the \textit{Krenn's conjecture}, since it remains unresolved to this day.\\

The problem introduces a unique form of vertex colourings in graphs that incorporates perfect matchings, and that was never investigated before.
The Krenn's conjecture, if solved, might reveal new intriguing insights in quantum optical physics.
In fact, Mario Krenn showed in his publication~\cite{Krenn_2017} a link between experimental setups for generating high-dimensional multipartite quantum states and a specific problem in graph theory.
While this preparatory work focuses only on the graph theory aspects and does not analyse the physical implications of the findings, the reader is encouraged to read the original article by Mario Krenn to acquire a deeper understanding of the physical aspects. \\

The problem is about proving the existence of a bound on some quantity called the \textit{weighted matching index} of a graph and is defined in details in the first chapter. In the context of this thesis, I will focus my research on finding at least some bound, constant if possible, on that value for special cases of the Krenn's conjecture. The approach I will take is based on the fact that I believe the conjecture to be true, but of course the discovery of a counter example can not be excluded. This preparatory work explains in details the Krenn's conjecture, establishes a state of the art in the domain, and defines a plan of realization of my research. At the end of this preparatory work, the goal will be to fully understand the problem and get a better view of what can be done to solve it.