\setcounter{secnumdepth}{-1}

\chapter{Conclusion}

In conclusion, the first goals that we fixed ourselves for that preparatory work are completed. At first, a proper definition of the Krenn's conjecture was established. For this, we showed an interesting parallel between monochromatic graphs and perfectly monochromatic graphs that was presented by Mario Krenn himself in \cite{wordpress}. A simplified version of the conjecture was explained, which was solved by Bogdanov in \cite{bogdanov}. Nevertheless, we could understand and redo the Bogdanov's proof to make it stick to our own notations. Keeping in mind the simplified version of the conjecture, we could define and understand all the concepts linked to the complete Krenn's conjecture, including the notions of perfectly monochromatic graphs and of weighted matching index. \\

Secondly, we could summarize the state of the research on the Krenn's conjecture up to this day. Thanks to Bogdanov in \cite{bogdanov}, the conjecture is already solved for real, positive weights. Also, Chandran and Gajjala proved in their paper \cite{chandran} the conjecture in the special case of graphs that have a matching index different from 1. In that same paper, they could find some upper bounds on the weighted matching index in term of minimum degree and of edge connectivity. We managed to redo these last proofs here in the context of our own work. \\

Lastly, the beginning of a plan of realization of our research was built. This plan includes the study of perfectly monochromatic graphs with a matching index of 1 and integer weights. We showed, as a first result, that the study of such graphs could be reduced to the study of graphs that have weights included in $\{-1, 1\}$. Furthermore, the hypothesis of implementing a program was analysed. Such a program would allow us to adopt a more experimental approach in the research by drawing random graphs constrained by our studied case. \\

To finish this preparatory work, I would like to thank again my promoters, Gwenaël Joret and Yelena Yuditsky, who helped me so much in my understanding of the different proofs we came across. Their help was highly appreciated and I had - and still will have - a very good experience working with them.